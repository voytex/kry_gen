\documentclass[titlepage]{article}

\usepackage[czech]{babel}
\usepackage{graphicx}
\usepackage{tikz}
\usepackage{hyperref}
\usepackage{dirtree}

\makeatletter
\tikzset{
    database/.style={
        path picture={
            \draw (0, 1.5*\database@segmentheight) circle [x radius=\database@radius,y radius=\database@aspectratio*\database@radius];
            \draw (-\database@radius, 0.5*\database@segmentheight) arc [start angle=180,end angle=360,x radius=\database@radius, y radius=\database@aspectratio*\database@radius];
            \draw (-\database@radius,-0.5*\database@segmentheight) arc [start angle=180,end angle=360,x radius=\database@radius, y radius=\database@aspectratio*\database@radius];
            \draw (-\database@radius,1.5*\database@segmentheight) -- ++(0,-3*\database@segmentheight) arc [start angle=180,end angle=360,x radius=\database@radius, y radius=\database@aspectratio*\database@radius] -- ++(0,3*\database@segmentheight);
        },
        minimum width=2*\database@radius + \pgflinewidth,
        minimum height=3*\database@segmentheight + 2*\database@aspectratio*\database@radius + \pgflinewidth,
    },
    database segment height/.store in=\database@segmentheight,
    database radius/.store in=\database@radius,
    database aspect ratio/.store in=\database@aspectratio,
    database segment height=0.1cm,
    database radius=0.25cm,
    database aspect ratio=0.35,
}
\makeatother



\title{Generátor úloh do aplikované kryptografie\\Kontrolní studie}
\author{Michal Homola,\\Dominik Chremčík,\\Jiří Marák,\\Vojtěch Lukáš}

\begin{document}
\maketitle

\tableofcontents

\section*{Úvod}
%TODO: Úvod, o projektu, autoři, kdo má co na práci apod.

\section{Architektura}
Schéma připravovaného systému lze vidět na obr.\,\ref{fig:sys}. Úlohy budou uloženy v SQL databázi. K~této databázi bude mít přístup pouze webový PHP~server. Ten slouží jako \uv{prostředník} mezi klientem a databází. Dále by měl do úloh vkládat náhodná data (klíče apod.) a~případně také vyhodnocovat výsledky. 
Klientská aplikace bude fungovat jako přístupový bod a sehrávat roli prezentační vrstvy. Pro jednoduchost bude vyvinuta v jazyce Python s oddělenou logickou vrstvou. Bude tedy možné na tuto vrstvu napojit i jednoduché grafické rozhraní. 
\begin{figure}[h!]
    \centering
    

\begin{tikzpicture}
    \draw (-.5,0) node[database, label=below: SQL databáze, database radius = .5cm, database segment height = .25cm]{};
    \draw[densely dotted, thick] (0,0) -- (1.5,0);
    \draw (1.5, -.7) rectangle (2.5, .7);
    \draw (1.5, .35) -- (2.5, .35);
    \draw (1.5, 0) -- (2.5, 0);
    \draw (1.5, -.35) -- (2.5, -.35);
    \draw (2, -.7) node[below]{.NET server};

    \draw[dashed] (-2,-1.5) rectangle (3.5,1.5) coordinate (box_right);
    \draw (.75, 1.5) node[above]{Microsoft Azure};
    \draw[dashed] (box_right) ++(down: 1.5) -- ++(right: 2) coordinate (pc_left); 
    \draw (pc_left) rectangle ++(1, .75);
    \draw (pc_left) -- ++(-.25, -.25) -- ++(right: .75) coordinate (pc_text) -- ++(right: .25) -- ++(.25, .25);
    \draw (pc_text) node[below]{Python klient};

\end{tikzpicture}    
    \caption{Schéma systému}
    \label{fig:sys}
\end{figure}




\subsection{Databáze úloh}
Databáze bude ve formátu...

\subsection{Back-end}
Architektura back-endu je navržena podle doporučení REST API. Původní návrhy řešení počítaly s využitím .NET serveru na portálu Microsoft Azure. Nakonec bylo ale upřednostněno řešení využívající PHP server. Celé řešení back-endu bylo založeno na \cite{restapi}. Od začátku byl projekt vyvíjen přímo na serveru pro usnadnění přístupu. 

\subsubsection{Soubory webového serveru}
\dirtree{%
    .1 web.
    .2 Model.
    .3 Database.php.
    .3 TaskModel.php.
    .2 Controller.
    .3 Api.
    .4 BaseController.php.
    .4 TaskController.php.
    .2 inc.
    .3 config.php.
    .3 bootstrap.php.
    .2 index.php.
}
\vspace{1em}
Moduly ve složce \texttt{Model} slouží k propojení a komunikaci s databází. Modul \texttt{Task\-Controller.php} je využit ke zpracování různých požadavků z uživatelské aplikace. Soubory ve složce \texttt{inc} by se daly označit jako \uv{servisní moduly}: \texttt{config.php} definuje přístupové údaje k databázi jako konstanty\footnote{Vzhledem k citlivé povaze obsažených informací je tento soubor vyřazen z verzování pomocí záznamu v \texttt{.gitignore}.}; modul \texttt{boot\-strap.php} pak obstarává správné propojení modulů. Modul \texttt{index.php} je pak ten, který přímo komunikuje s klientskou aplikací, třídí její požadavky a korektně na ně odpovídá. 

\subsection{Front-end}

\section{Současný stav}

\section*{Závěr}

\begin{thebibliography}{9}
    \bibitem{restapi}
    SONI, Sajal. How to build a simple REST API in PHP. \emph{En\-va\-to Tuts+} [on\-li\-ne]. 27-5-2021 [cit. 2023-03-25]. Dostupné z: \url{https://code.tutsplus.com/tutorials/how-to-build-a-simple-rest-api-in-php--cms-37000}
\end{thebibliography}

\end{document}