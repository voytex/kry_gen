% ----------------------------------------------------------------------
% Authors:     Tomas Fryza, Wykys
%              Dept. of Radio Electronics, Brno Univ. of Technology
% Created:     2018-05-22
% Last update: 2019-07-24
% Description: Template using official colors of Brno University of 
% Technology defined by graphical manual in 2015.
% ----------------------------------------------------------------------
% License:     CC BY-NC-SA 4.0
% Keep the name of the authors. You may not use the template for 
% commercial purposes. Share and modify as you like under the same 
% license as the original. 
%
% https://creativecommons.org/licenses/by-nc-sa/4.0/
% ----------------------------------------------------------------------

% ----------------------------------------------------------------------
% PACKAGES
% ----------------------------------------------------------------------

%% Sets aspect ratio to 16:10, and frame size to 160mm by 100mm
% Please, do not use old-school 4:3 ratio anymore:)
\documentclass[aspectratio=1610]{beamer}

%% Select your favorite language
\usepackage[english]{babel} % Multilingual support for LaTeX
% \usepackage[czech]{babel}

\usepackage[utf8]{inputenc} % Accept different input encodings
\usepackage{ucs} % Extended UTF-8 input encoding support for LaTeX
\usepackage{graphicx} % Enhanced support for graphics
\usepackage{listings} % Typeset source code listings using LaTeX
\usepackage{color} % Colour control for LaTeX documents

%% Copy your favorite logo from "vut_logo_archive/" to root folder 
% and rename file to "logo.png"

%% Select color theme
%\usepackage{themevut} % All in red:)
% \usepackage[FA]{themevut}
% \usepackage[FAST]{themevut}
% \usepackage[FaVU]{themevut}
% \usepackage[FCH]{themevut}
\usepackage[FEKT]{themevut}
% \usepackage[FIT]{themevut}
% \usepackage[FP]{themevut}
% \usepackage[FSI]{themevut}
% \usepackage[CESA]{themevut}
% \usepackage[USI]{themevut}

% ----------------------------------------------------------------------
% TITLE PAGE
% ----------------------------------------------------------------------

% The short title appears at the bottom of every slide, the full title
% is only on the title page.
\title[Generátor úloh do aplikované kryptografie]
{Generátor úloh do aplikované kryptografie}

% Type of project, i.e. Bachelor, Master, PhD, etc.
% \subtitle
% {Type of project}

% Your name
\author[ Homola, Chrenčík, Marák, Lukáš]
{Michal Homola, Dominik Chrenčík, Jiří Marák, Vojtěch Lukáš}

% Your institution
\institute
{MPC-KRY \\
Ústav telekomunikací  \\
VUT v Brně
}

% Date, can be changed to a custom date
\date{\today}

% Logo on title page
\titlegraphic{\includegraphics[height=.15\textheight]{logo.png}}

\begin{document}

% ----------------------------------------------------------------------
% PRESENTATION SLIDES
% ----------------------------------------------------------------------

\begin{frame}
    % Print the title page as the first slide
    \titlepage
\end{frame}

% ----------------------------------------------------------------------

\begin{frame}{Zadání}
    \begin{itemize}
        \item Navrhněte a implementujte vlastní službu pro generování úloh do aplikované
kryptografie.
        \item 
    
    \end{itemize}
\end{frame}
% ----------------------------------------------------------------------

% https://cs.overleaf.com/learn/latex/Lists
\begin{frame}{Rešení}
    \textbf{Unordered lists}
    \begin{itemize}
        \item Lorem ipsum dolor sit amet, consectetuer adipiscing elit.
        \item Etiam sapien elit, consequat eget, tristique non, venenatis quis, ante.
        \item Aliquam erat volutpat.
        \item Integer lacinia.
        \item Cras pede libero, dapibus nec, pretium sit amet, tempor quis.
    \end{itemize}
    \bigskip % Vertical space

    \textbf{Ordered lists}
    \begin{enumerate}
        \item \alert{Lorem ipsum dolor} sit amet, consectetuer adipiscing elit.
        \item Etiam sapien elit, consequat eget, tristique non, venenatis quis, ante.
        \item Aliquam erat volutpat:
        \begin{itemize}
            \item Integer lacinia.
            \item Cras pede libero, dapibus nec, pretium sit amet, tempor quis.
        \end{itemize}
    \end{enumerate}
\end{frame}

% ----------------------------------------------------------------------

\begin{frame}{Text}
    \begin{columns}
        % Left column and width
        \column{0.45\textwidth}
        \textit{Lorem ipsum} dolor sit amet, consectetuer adipiscing elit. Etiam sapien elit, consequat eget, tristique non, venenatis quis, ante. Duis sapien nunc, commodo et, interdum suscipit, sollicitudin et, dolor.
        \bigskip
        
        \texttt{Fusce tellus.} Praesent in mauris eu tortor porttitor accumsan. Nullam feugiat, turpis at pulvinar vulputate, erat libero tristique tellus, nec bibendum odio risus sit amet ante. Vestibulum fermentum tortor id mi.

        % Right column and width
        \column{0.45\textwidth}
        \textbf{Lorem ipsum} dolor sit amet, consectetuer adipiscing elit. Etiam sapien elit, consequat eget, tristique non, venenatis quis, ante. Duis sapien nunc, commodo et, interdum suscipit, sollicitudin et, dolor.
        \bigskip
        
        Fusce tellus. Praesent in mauris eu tortor porttitor accumsan. Nullam feugiat, turpis at pulvinar vulputate, erat libero tristique tellus, nec bibendum odio risus sit amet ante. Vestibulum fermentum tortor id mi.
    \end{columns}
\end{frame}

% ----------------------------------------------------------------------
% https://en.wikibooks.org/wiki/LaTeX/Floats,_Figures_and_Captions
\begin{frame}{?Diagram}
    \begin{center}
        \begin{figure}
            \includegraphics[width=0.4\textwidth]{logo.png}
            \caption{Your caption}
        \end{figure}
    \end{center}
\end{frame}

% ----------------------------------------------------------------------
% https://en.wikibooks.org/wiki/LaTeX/Tables
\begin{frame}{Tabulka}
    \begin{center}
        \begin{table}
            \caption{Your caption}
            \begin{tabular}{l | c | c | c | r}
                \textbf{Function name} & \textbf{Duration} & \textbf{Complexity} & \textbf{Length} & \textbf{Score}\\
                \hline \hline
                Algo 1 & 0.0159 & 0.50 & 125 & 78 \\
                Algo 2 & 0.0453 & 0.65 & 854 & 88 \\
                Algo 3 & 0.8642 & 0.77 &  84 & 95 \\
                Algo 4 & 0.0020 & 0.24 & 638 & 76 \\
            \end{tabular}
        \end{table}
    \end{center}
\end{frame}

% ----------------------------------------------------------------------
% https://en.wikibooks.org/wiki/LaTeX/Mathematics
\begin{frame}{Rovnice}
    Pythagorean theorem can be written in one short equation as: $a^2 + b^2 = c^2$ where $c$ is the longest side of the triangle, $a$ and $b$ are the other two sides.

    \vfill % Rubber length which can stretch or shrink vertically

    Other useful equations (thank you \textit{John Napier}):
    \begin{equation}
        \log_b (x\cdot y) = \log_b (x) + \log_b (y)
    \end{equation}
    \begin{equation}
        \log_b \left( \frac{x}{y} \right) = \log_b (x) - \log_b (y)
    \end{equation}
    \begin{equation}
        \log_b (x^p) = p\cdot \log_b (x)
    \end{equation}
    \begin{eqnarray}
        \log_b(x) = y & \text{exactly if} & b^y = x\
    \end{eqnarray}
\end{frame}

% ----------------------------------------------------------------------
% https://cs.overleaf.com/learn/latex/Code_listing
\begin{frame}[fragile]{Příklad kódu}
    % Need to use the "fragile" option when verbatim is used in the slide

\begin{lstlisting}[language=python,title={Příklad Python kódu}]

while True:
    valid_codes = print_all_tasks()
    code = str(input(f"{C_BLUE}[Skore: {SCORE}] {C_YELLOW}Zadejte kod ulohy, kterou si prejete resit:{C_RES}"))
    if code not in valid_codes:
        print(f"{C_RED}spatny kod{C_RES}")
    else:    
        clear_console()
        request = requests.get(f"{API}/task?code={code}")
\end{lstlisting}
    \vfill

\begin{lstlisting}[language=php,title={Příklad PHP kódu}]
$random = rand(1,2); //slouzi k vyberu prvocislo / slozene cislo
//podle vyberu se operand $prime nastavi na True/False
if ($random == 1) {
//prvocislo
$X = rand(530, 10000);
$C = gmp_nextprime($X); //vysledek

$prime= True;

} else {
//slozene cislo
\end{lstlisting}


\end{frame}

% ----------------------------------------------------------------------
% Remind the main results at the end of your presentation
\begin{frame}{Dosažené výsledky}
    \begin{itemize}
        \item Funkční generátor
    \end{itemize}
\end{frame}

% ----------------------------------------------------------------------

% ----------------------------------------------------------------------

\end{document}
