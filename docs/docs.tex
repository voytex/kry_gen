\documentclass[titlepage]{article}

\usepackage[czech]{babel}
\usepackage{graphicx}
\usepackage{tikz}
\usepackage{hyperref}
\usepackage{dirtree}
\usepackage{amsmath}

\input{../control_study/schemes/database-glyph.tex}

\title{Generátor úloh do aplikované kryptografie\\Kontrolní studie}
\author{Michal Homola,\\Dominik Chrenčík,\\Jiří Marák,\\Vojtěch Lukáš}

\begin{document}
\maketitle

\tableofcontents

\section*{Úvod}
\addcontentsline{toc}{section}{Úvod}
Předmětem této dokumentace je představit vizi projektu s názvem \uv{Generátor kryptografických úloh}. První část bude věnována teoretickému popisu systému jako celku. \dots

\section{Architektura}
Schéma připravovaného systému lze vidět na obr.\,\ref{fig:sys}. Úlohy budou uloženy v SQL databázi. K~této databázi bude mít přístup pouze webový PHP~server. Ten slouží jako \uv{prostředník} mezi klientem a databází. Dále by měl do úloh vkládat náhodná data (klíče apod.) a~případně také vyhodnocovat výsledky. 
Klientská aplikace bude fungovat jako přístupový bod a sehrávat roli prezentační vrstvy. Pro jednoduchost bude vyvinuta v jazyce Python s oddělenou logickou vrstvou. Bude tedy možné na tuto vrstvu napojit i jednoduché grafické rozhraní. 
\begin{figure}[h!]
    \centering
    \include{../control_study/schemes/system-scheme.tex}    
    \caption{Schéma systému}
    \label{fig:sys}
\end{figure}

\end{document}